%%%%%%%%%%%%%%%%%%%%%%%%%%%%%%%%%%%%%%%%%
% Focus Beamer Presentation
% LaTeX Template
% Version 1.0 (8/8/18)
%
% This template has been downloaded from:
% http://www.LaTeXTemplates.com
%
% Original author:
% Pasquale Africa (https://github.com/elauksap/focus-beamertheme) with modifications by
% Vel (vel@LaTeXTemplates.com)
%
% Template license:
% GNU GPL v3.0 License
%
% Important note:
% The bibliography/references need to be compiled with bibtex.
%
%%%%%%%%%%%%%%%%%%%%%%%%%%%%%%%%%%%%%%%%%

%----------------------------------------------------------------------------------------
%    PACKAGES AND OTHER DOCUMENT CONFIGURATIONS
%----------------------------------------------------------------------------------------

\documentclass{beamer}

\usetheme{focus} % Use the Focus theme supplied with the template
% Add option [numbering=none] to disable the footer progress bar
% Add option [numbering=fullbar] to show the footer progress bar as always full with a slide count

% Uncomment to enable the ice-blue theme
%\definecolor{main}{RGB}{92, 138, 168}
%\definecolor{background}{RGB}{240, 247, 255}

%------------------------------------------------

\usepackage{booktabs} % Required for better table rules
\usepackage{listings}

\lstset{
    escapeinside=||
}

\newcommand{\todo}[1]{\color{red}{TODO: #1}}

%----------------------------------------------------------------------------------------
%     TITLE SLIDE
%----------------------------------------------------------------------------------------

\title{Introduction to Clang/LLVM}
%\subtitle{Subtitle}
\author{Christian Sharpsten}
\titlegraphic{\includegraphics[scale=0.6]{images/llvm_logo_derivative.png}}
%\institute{\todo{\\ Institute Name \\ Institute Address}}
\date{01 NOV 2019}

%------------------------------------------------

\begin{document}

%------------------------------------------------

\begin{frame}
    \maketitle
\end{frame}

\begin{frame}{Outline}
    \begin{itemize}
        \item Background
        \begin{itemize}
            \item About LLVM
            \item General Compilation Process
        \end{itemize}
        \vspace{1ex}

        \item LLVM Compilation Process
        \vspace{1ex}

        \item Writing an Optimization Pass
        \begin{itemize}
            \item Building LLVM
            \item A Simple Pass
            \item Taint Tracing to Detect Vulnerable memcpys
            \item Control-Flow Obfuscation
        \end{itemize}
    \end{itemize}
\end{frame}

%----------------------------------------------------------------------------------------
%     SECTION 1
%----------------------------------------------------------------------------------------

\section{Background} % Section title slide, unnumbered

%------------------------------------------------

\begin{frame}{General Compilation Process}

\end{frame}

\begin{frame}{General Compilation Process (Lexing/Parsing)}
SSA/AST
\end{frame}

\begin{frame}{General Compilation Process (Code Generation)}
Three-address code
\end{frame}

%------------------------------------------------

\begin{frame}{What is LLVM?}
    LLVM is a ``collection of modular and reusable compiler and toolchain technologies.'' \cite{llvm_org}
    \vspace{1em}

    LLVM is composed of multiple sub-projects including:
    {\footnotesize
    \begin{enumerate}
        \item \textbf{LLVM Core} - A set of libraries implementing an optimizer and code generators for common CPUs
        % AArch64, AMDGPU, ARM, BPF, Hexagon, Mips, MSP430, NVPTX, PowerPC, Sparc, SystemZ, X86, XCore
        \item \textbf{Clang} - A front-end compiler
        \item \textbf{LLDB} - A native debugger
        \item \textbf{libc++} - A C++14 compliant STL
        \item \textbf{compiler-rt} - Compiler run-time libraries (intrinsics, ASAN, TSAN, MSAN, etc.)
        % ASAN - Detect OOB accesses, UAF, double free, etc.
        % TSAN - Detect race conditions (experimental)
        % MSAN - Detect uninitialized reads
        \item \textbf{klee} - A symbolic executor
        \item \textbf{LLD} - A drop-in replacement for system linkers such as \texttt{ld}
    \end{enumerate}
    }
\end{frame}

\begin{frame}{Why LLVM?}
    Composable

    Easy to hack on

    LLVM JIT

    Cross-compiler by default (can build in all backends at once instead of single builds like gcc)

    Keystone/Capstone/McSema/Other tools
\end{frame}

\begin{frame}{LLVM Core Tools}
    opt

    llvm-dis

    llvm-as

    llvm-mc

    lto library?
\end{frame}

%----------------------------------------------------------------------------------------
%     SECTION 2
%----------------------------------------------------------------------------------------

\section{LLVM Compilation Process}

%------------------------------------------------

\begin{frame}{LLVM Sub-Projects}
\end{frame}

%------------------------------------------------

\begin{frame}{LLVM Tools}
General overview
List of llvm tools and how they align with the compilation process
\end{frame}

%------------------------------------------------

\begin{frame}{Clang: Pre-Processing}
\end{frame}

%------------------------------------------------

\begin{frame}{Clang: Lexing}
DFA
\end{frame}

%------------------------------------------------

\begin{frame}{Clang: Parsing}
AST
Bitcode generation
\end{frame}

%------------------------------------------------

\begin{frame}{Opt: Optimization}
Bitcode parsing
How to view passes?
How to run one pass?
Sampling of some important ones (SROA, DCE, etc)
Module/Function/BB passes
\end{frame}

%------------------------------------------------

\begin{frame}{Llc: Code Generation}
Tablegen
DAG and Legalization
Register allocation
Arch-specific optimizations?
\end{frame}

%----------------------------------------------------------------------------------------
%     SECTION 3
%----------------------------------------------------------------------------------------

\section{Writing an Optimization Pass}

%------------------------------------------------

\begin{frame}{Building LLVM}
    \begin{itemize}
        \item LLVM switched from svn to a single git monorepo as of 21 OCT 2019 (exciting!)
        \item LLVM uses CMake. You can control the build in a number of ways:
        \begin{itemize}
            \item Generator (Ninja, Unix Makefiles, VS, Xcode)
            \item Build type (Debug, Release, RelWithDebInfo, MinSizeRel)
            \item Enabled sub-projects (test suite, libcxx, lldb, lld, etc.)
            \item Backend targets (X86, Mips, PowerPC, etc.)
        \end{itemize}
        \item Depending on which features you enable, LLVM can take a long time to compile.
    \end{itemize}
\end{frame}

%------------------------------------------------

\begin{frame}[fragile]{Building LLVM}
    For this exercise, we only need to build Clang and the X86 backend.

    \begin{lstlisting}[gobble=4]
    $ git clone https://github.com/llvm/llvm-project.git |\pause|
    $ cd llvm-project && git checkout llvmorg-9.0.0 |\pause|
    $ mkdir build && cd build |\pause|
    $ cmake -G Ninja                    \
        -DLLVM_ENABLE_PROJECTS='clang'  \
        -DCMAKE_BUILD_TYPE=Debug        \
        -DLLVM_TARGETS_TO_BUILD=X86
        ../llvm |\pause|
    $ time ninja

    \end{lstlisting}
\end{frame}

%------------------------------------------------

\begin{frame}{Building LLVM}
    LLVM can take a while to compile...

    \begin{table}
        \footnotesize
        \centering
        \begin{tabular}{l c c c | c c}
            \toprule
            Generator & Build Type & Sub-Projects & Targets & Time (m) & Size (GB) \\
            \toprule
            \textbf{Ninja} & \textbf{Debug} & \textbf{Clang} & \textbf{X86} & \textbf{120.15} & \textbf{44.0} \\
            Ninja      & Release & Clang & X86 & 75.03  & 1.7  \\
            Ninja      & Debug   & Clang & All & 205.65 & 59.5 \\
            Ninja      & Release & Clang & All & 106.33 & 2.5  \\
            Make       & Release & Clang & X86 & 433.30 & 1.8  \\
            Make (-j8) & Release & Clang & X86 & 77.13  & 1.7  \\
            \bottomrule
        \end{tabular}
        \caption{\footnotesize LLVM Compile Time Benchmarks (Ubuntu 18.04 VM, 6 cores, 16GB RAM)}
    \end{table}
    % Commands used to benchmark:
    %   cmake -G Ninja -DLLVM_ENABLE_PROJECTS='clang' -DCMAKE_BUILD_TYPE=Debug -DLLVM_PROJECTS_TO_BUILD=X86 ../llvm && time ninja
    %   cmake -G Ninja -DLLVM_ENABLE_PROJECTS='clang' -DCMAKE_BUILD_TYPE=Release -DLLVM_PROJECTS_TO_BUILD=X86 ../llvm && time ninja
    %   cmake -G Ninja -DLLVM_ENABLE_PROJECTS='clang' -DCMAKE_BUILD_TYPE=Debug ../llvm && time ninja
    %   cmake -G Ninja -DLLVM_ENABLE_PROJECTS='clang' -DCMAKE_BUILD_TYPE=Release ../llvm && time ninja
    %   cmake -G 'Unix Makefiles' -DLLVM_ENABLE_PROJECTS='clang' -DCMAKE_BUILD_TYPE=Debug -DLLVM_TARGETS_TO_BUILD=X86 ../llvm && time make
    %   cmake -G 'Unix Makefiles' -DLLVM_ENABLE_PROJECTS='clang' -DCMAKE_BUILD_TYPE=Debug -DLLVM_TARGETS_TO_BUILD=X86 ../llvm && time make -j8

    Keep in mind that debug artifacts can be quite large as well.
\end{frame}

\begin{frame}[fragile]{Building LLVM}
    \begin{alertblock}{High Memory Usage}
        Watch out for out-of-memory errors when linking. Restart ninja/make with less threads if a link process is killed.
    \end{alertblock}

    \vspace{-2ex}

    \begin{overprint}
        \onslide<1>
        \begin{lstlisting}[basicstyle=\footnotesize,gobble=8]
        [2361/2742] Linking CXX shared module
            lib/CheckerOptionHandlingAnalyzerPlugin.so
        FAILED: lib/CheckerOptionHandlingAnalyzerPlugin.so
        : && /usr/bin/c++ -fPIC -fPIC ...
        ...
        collect2: fatal error: ld terminated with signal 9 [Killed]
        compilation terminated.
        ninja: build stopped: subcommand failed.
        \end{lstlisting}

        \onslide<2>
        \begin{lstlisting}[basicstyle=\footnotesize,gobble=8]
        [2359/2742] Linking CXX executable bin/clang-diff
        FAILED: bin/clang-diff
        : && /usr/bin/c++  -fPIC -fvisibility-inlines-hidden ...
        ...
        /usr/bin/ld: BFD (GNU Binutils for Ubuntu) 2.30 internal error,
            aborting at ../../bfd/merge.c:908 in
            _bfd_merged_section_offset

        /usr/bin/ld: Please report this bug.

        collect2: error: ld returned 1 exit status
        ninja: build stopped: subcommand failed.
        \end{lstlisting}

        \vspace{-2ex}
        \rule{3cm}{0.5pt}

        {\footnotesize \url{https://bugs.debian.org/cgi-bin/bugreport.cgi?bug=874674}}
    \end{overprint}
\end{frame}

%------------------------------------------------

\begin{frame}[fragile]{LLVM IR}
    LLVM comes with IR and tablegen syntax highlighting for vim, emacs, and vscode, among other editors.

    \vspace{1cm}

    \begin{lstlisting}[basicstyle=\footnotesize,gobble=4]
    $ cd ~/.vim
    $ ln -s ~/llvm-project/llvm/utils/vim/
            {ftdetect,ftplugin,indent,syntax}
    \end{lstlisting}
\end{frame}

\begin{frame}{LLVM IR}
    use/def
    Everything is a value and a use
    CallInst vs InvokeInst (exception handling)
\end{frame}

\begin{frame}{LLVM IR}
    C code for a short sample
\end{frame}

\begin{frame}{LLVM IR}
    IR for the same sample
\end{frame}

%------------------------------------------------

\begin{frame}{Writing a Pass (Hello World)}
    Module vs Function vs BasicBlock pass
    LoopPass RegionPass and MachineFunctionPass
\end{frame}

\begin{frame}{Writing a Pass (Hello World)}
    Hello world pass
\end{frame}

\begin{frame}{Writing a Pass}
    Builder API
\end{frame}

\begin{frame}{Writing a Pass}
    \todo{Any other interesting things?}
\end{frame}

%------------------------------------------------

\begin{frame}{Writing a Pass (\_\_\_\_)}
    \todo{Bogus arguments? Constant obfuscation? Detect vuln strcpy? Taint tracing?}
\end{frame}

%------------------------------------------------

\begin{frame}{ExtractBB Pass}
    Liveness analysis
\end{frame}

\begin{frame}{ExtractBB Pass}
    Demo with IDA screenshots
\end{frame}

%------------------------------------------------

\begin{frame}{Columns}
    \begin{columns}
        \column{0.5\textwidth}
            This text appears in the left column and wraps neatly with a margin between columns.

        \column{0.5\textwidth}
            \includegraphics[width=\linewidth]{Images/placeholder.jpg}
    \end{columns}
\end{frame}

%------------------------------------------------

\begin{frame}{Lists}
    \begin{columns}[T, onlytextwidth] % T for top align, onlytextwidth to suppress the margin between columns
        \column{0.33\textwidth}
            Items:
            \begin{itemize}
                \item Item 1
                \begin{itemize}
                    \item Subitem 1.1
                    \item Subitem 1.2
                \end{itemize}
                \item Item 2
                \item Item 3
            \end{itemize}

        \column{0.33\textwidth}
            Enumerations:
            \begin{enumerate}
                \item First
                \item Second
                \begin{enumerate}
                    \item Sub-first
                    \item Sub-second
                \end{enumerate}
                \item Third
            \end{enumerate}

        \column{0.33\textwidth}
            Descriptions:
            \begin{description}
                \item[First] Yes.
                \item[Second] No.
            \end{description}
    \end{columns}
\end{frame}

%------------------------------------------------

\begin{frame}[focus]
    Questions?
\end{frame}

%----------------------------------------------------------------------------------------
%     CLOSING/SUPPLEMENTARY SLIDES
%----------------------------------------------------------------------------------------

\appendix

\begin{frame}{References}
    \nocite{*} % Display all references regardless of if they were cited
    \bibliography{references.bib}
    \bibliographystyle{plain}
\end{frame}

%------------------------------------------------

\begin{frame}{Backup Slide}
    This is a backup slide, useful to include additional materials to answer questions from the audience.
    \vfill
    The package \texttt{appendixnumberbeamer} is used to refrain from numbering appendix slides.
\end{frame}

%----------------------------------------------------------------------------------------

\end{document}
