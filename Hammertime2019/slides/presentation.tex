%%%%%%%%%%%%%%%%%%%%%%%%%%%%%%%%%%%%%%%%%
% Focus Beamer Presentation
% LaTeX Template
% Version 1.0 (8/8/18)
%
% This template has been downloaded from:
% http://www.LaTeXTemplates.com
%
% Original author:
% Pasquale Africa (https://github.com/elauksap/focus-beamertheme) with modifications by
% Vel (vel@LaTeXTemplates.com)
%
% Template license:
% GNU GPL v3.0 License
%
% Important note:
% The bibliography/references need to be compiled with bibtex.
%
%%%%%%%%%%%%%%%%%%%%%%%%%%%%%%%%%%%%%%%%%

%----------------------------------------------------------------------------------------
%    PACKAGES AND OTHER DOCUMENT CONFIGURATIONS
%----------------------------------------------------------------------------------------

\documentclass{beamer}

\usetheme{focus} % Use the Focus theme supplied with the template
% Add option [numbering=none] to disable the footer progress bar
% Add option [numbering=fullbar] to show the footer progress bar as always full with a slide count

% Uncomment to enable the ice-blue theme
%\definecolor{main}{RGB}{92, 138, 168}
%\definecolor{background}{RGB}{240, 247, 255}

%------------------------------------------------

\usepackage{booktabs} % Required for better table rules
\usepackage{listings}

\lstset{
    escapeinside=||
}

\newcommand{\todo}[1]{\color{red}{TODO: #1}}

%----------------------------------------------------------------------------------------
%     TITLE SLIDE
%----------------------------------------------------------------------------------------

\title{Introduction to Clang/LLVM}

%\subtitle{Subtitle}

\author{Christian Sharpsten}

\titlegraphic{\includegraphics[scale=0.6]{images/llvm_logo_derivative.png}} % Optional title page image, comment this line to remove it

\institute{\todo{\\ Institute Name \\ Institute Address}}

\date{01 NOV 2019}

%------------------------------------------------

\begin{document}

%------------------------------------------------

\begin{frame}
    \maketitle % Automatically created using the information in the commands above
\end{frame}

\begin{frame}{Outline}
    \begin{itemize}
        \item Background
        \item LLVM Compilation Process
        \item Writing an Optimization Pass
    \end{itemize}
\end{frame}

%----------------------------------------------------------------------------------------
%     SECTION 1
%----------------------------------------------------------------------------------------

\section{Background} % Section title slide, unnumbered

%------------------------------------------------

\begin{frame}{What is LLVM?}
    This is a simple slide.
    Keystone/Capstone/McSema/Other tools
\end{frame}

%------------------------------------------------

\begin{frame}{Why LLVM?}

\end{frame}

%------------------------------------------------

\begin{frame}{General Compilation Process}

\end{frame}

%------------------------------------------------

\begin{frame}{General Compilation Process (Lexing/Parsing)}
    This slide is not numbered and is citing reference \cite{knuth74}.
    SSA/AST
\end{frame}

%------------------------------------------------

\begin{frame}{General Compilation Process (Code Generation)}
    This slide is not numbered and is citing reference \cite{knuth74}.
    Three-address code
\end{frame}

%------------------------------------------------

\begin{frame}{Typesetting and Math}
    The packages \texttt{inputenc} and \texttt{FiraSans}\footnote{\url{https://fonts.google.com/specimen/Fira+Sans}}\textsuperscript{,}\footnote{\url{http://mozilla.github.io/Fira/}} are used to properly set the main fonts.
    \vfill
    This theme provides styling commands to typeset \emph{emphasized}, \alert{alerted}, \textbf{bold}, \textcolor{example}{example text}, \dots
    \vfill
    \texttt{FiraSans} also provides support for mathematical symbols:
    \begin{equation*}
        e^{i\pi} + 1 = 0.
    \end{equation*}
\end{frame}

%----------------------------------------------------------------------------------------
%     SECTION 2
%----------------------------------------------------------------------------------------

\section{LLVM Compilation Process}

%------------------------------------------------

\begin{frame}{LLVM Sub-Projects}
\end{frame}

%------------------------------------------------

\begin{frame}{LLVM Tools}
General overview
List of llvm tools and how they align with the compilation process
\end{frame}

%------------------------------------------------

\begin{frame}{Clang: Pre-Processing}
\end{frame}

%------------------------------------------------

\begin{frame}{Clang: Lexing}
DFA
\end{frame}

%------------------------------------------------

\begin{frame}{Clang: Parsing}
AST
Bitcode generation
\end{frame}

%------------------------------------------------

\begin{frame}{Opt: Optimization}
Bitcode parsing
How to view passes?
How to run one pass?
Sampling of some important ones (SROA, DCE, etc)
Module/Function/BB passes
\end{frame}

%------------------------------------------------

\begin{frame}{Llc: Code Generation}
Tablegen
DAG and Legalization
Register allocation
Arch-specific optimizations?
\end{frame}

%----------------------------------------------------------------------------------------
%     SECTION 3
%----------------------------------------------------------------------------------------

\section{Writing an Optimization Pass}

%------------------------------------------------

\begin{frame}{Building LLVM}
    \begin{itemize}
        \item LLVM switched from svn to a single git monorepo as of 21 OCT 2019 (exciting!)
        \item LLVM uses CMake. You can control the build in a number of ways:
        \begin{itemize}
            \item Generator (Ninja, Unix Makefiles, VS, Xcode)
            \item Build type (Debug, Release, RelWithDebInfo, MinSizeRel)
            \item Enabled sub-projects (test suite, libcxx, lldb, lld, etc.)
            \item Backend targets (X86, Mips, PowerPC, etc.)
        \end{itemize}
        \item Depending on which features you enable, LLVM can take a long time to compile.
    \end{itemize}
\end{frame}

%------------------------------------------------

\begin{frame}[fragile]{Building LLVM}
    For this exercise, we only need to build Clang and the X86 backend.

    \begin{lstlisting}[gobble=4]
    $ git clone https://github.com/llvm/llvm-project.git |\pause|
    $ cd llvm-project && git checkout llvmorg-9.0.0 |\pause|
    $ mkdir build && cd build |\pause|
    $ cmake -G Ninja                    \
        -DLLVM_ENABLE_PROJECTS='clang'  \
        -DCMAKE_BUILD_TYPE=Debug        \
        -DLLVM_TARGETS_TO_BUILD=X86
        ../llvm |\pause|
    $ time ninja

    \end{lstlisting}
\end{frame}

%------------------------------------------------

\begin{frame}{Building LLVM}
LLVM can take a while to compile...

    \begin{table}
        \centering % Centre the table on the slide
        \begin{tabular}{l c c c c}
            \toprule
            Generator & Build Type & Sub-Projects & Targets & Time (m) \\
            \toprule
            \textbf{Ninja} & \textbf{Debug} & \textbf{Clang} & \textbf{X86} & \textbf{120.15} \\
            Ninja & Release & Clang & X86 & 0.00 \\
            Ninja & Debug & Clang & All & 0.00 \\
            Ninja & Release & Clang & All & 0.00 \\
            Make & Debug & Clang & X86 & 0.00 \\
            Make (-j12) & Debug & Clang & X86 & 0.00 \\
            \bottomrule
        \end{tabular}
        \caption{LLVM Compile Time Benchmarks (6-core, 16GB RAM Ubunut 18.04 VM)}
    \end{table}
\end{frame}

%------------------------------------------------

\begin{frame}{LLVM IR}
    Intro to some of the basic instructions
\end{frame}

\begin{frame}{LLVM IR}
    C code for a short sample
\end{frame}

\begin{frame}{LLVM IR}
    IR for the same sample
\end{frame}

%------------------------------------------------

\begin{frame}{Writing a Pass}
    Skeleton Code for a pass
    \todo{Module/function/bb pass?}
\end{frame}

\begin{frame}{Writing a Pass}
    Hello world pass
\end{frame}

%------------------------------------------------

\begin{frame}{A More Complex Pass}
    \todo{Bogus arguments? Constant obfuscation? Detect vuln strcpy? Taint tracing?}
\end{frame}

%------------------------------------------------

\begin{frame}{ExtractBB Pass}
    Liveness analysis
\end{frame}

\begin{frame}{ExtractBB Pass}
    Demo with IDA screenshots
\end{frame}

%------------------------------------------------

\begin{frame}{Columns}
    \begin{columns}
        \column{0.5\textwidth}
            This text appears in the left column and wraps neatly with a margin between columns.

        \column{0.5\textwidth}
            \includegraphics[width=\linewidth]{Images/placeholder.jpg}
    \end{columns}
\end{frame}

%------------------------------------------------

\begin{frame}{Lists}
    \begin{columns}[T, onlytextwidth] % T for top align, onlytextwidth to suppress the margin between columns
        \column{0.33\textwidth}
            Items:
            \begin{itemize}
                \item Item 1
                \begin{itemize}
                    \item Subitem 1.1
                    \item Subitem 1.2
                \end{itemize}
                \item Item 2
                \item Item 3
            \end{itemize}

        \column{0.33\textwidth}
            Enumerations:
            \begin{enumerate}
                \item First
                \item Second
                \begin{enumerate}
                    \item Sub-first
                    \item Sub-second
                \end{enumerate}
                \item Third
            \end{enumerate}

        \column{0.33\textwidth}
            Descriptions:
            \begin{description}
                \item[First] Yes.
                \item[Second] No.
            \end{description}
    \end{columns}
\end{frame}

%------------------------------------------------

\begin{frame}[focus]
    Thanks for using \textbf{Focus}!
\end{frame}

%----------------------------------------------------------------------------------------
%     CLOSING/SUPPLEMENTARY SLIDES
%----------------------------------------------------------------------------------------

\appendix

\begin{frame}{References}
    \nocite{*} % Display all references regardless of if they were cited
    \bibliography{references.bib}
    \bibliographystyle{plain}
\end{frame}

%------------------------------------------------

\begin{frame}{Backup Slide}
    This is a backup slide, useful to include additional materials to answer questions from the audience.
    \vfill
    The package \texttt{appendixnumberbeamer} is used to refrain from numbering appendix slides.
\end{frame}

%----------------------------------------------------------------------------------------

\end{document}
